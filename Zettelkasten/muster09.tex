\documentclass[11pt]{amsart}
\usepackage[utf8]{inputenc}
\usepackage{amssymb,amsmath}
\usepackage{verbatim}
\usepackage{color}
\usepackage{geometry}
\geometry{a4paper,left=2cm,right=2cm, top=1.5cm, bottom=1.5cm} 
\usepackage{amsthm}
\usepackage{stmaryrd}
\usepackage{graphicx}

%\includegraphics{?} setzt bild ein
%\ref{labelname} erstellt link zu labelname
%\label{labelname} kann einfach irgendwo drangesetz werden

\newtheorem{defi}{Definition}
\newtheorem{axiom}{Axiom}
\newtheorem{nota}{Notation}
\newtheorem{prop}{Proposition}
\newtheorem{satz}{Satz}
\newtheorem{umf}{Umformung}

\newenvironment{beweis}{\par\begingroup%
\settowidth{\leftskip}{\textsc{Beweis:~}}%
\noindent\llap{\textsc{Beweis:~}}}{\hfill$\Box$\par\endgroup}

\renewcommand{\baselinestretch}{1}
\newcommand{\words}{\Sigma^* \backslash \{\epsilon\}}
\newcommand{\etrans}[1]{\bar{\delta}(#1)}
\renewcommand{\P}{\mathbb{P}}

\title{Zettel 9 Lösung}
\author{Florian Lerch(2404605)}
%\date{} % Activate to display a given date or no date (if empty),
% otherwise the current date is printed 

\begin{document}
\subsection{31}
\begin{equation}
  A = \mathcal{P}(M), X_i \subset M, i \in I
  \mbox{ Berechne: } sup_{i\in I}X_i \mbox{ und } inf_{i\in I}X_i
  sup X_i = \cup_{i\in I}X_i 
  X_i \leq \cup_{i\in I}X_i \mbox{ Sei } B \subset M 
  \mbox{ eine Menge mit } X_i \leq B \forall i \in I 
  d.h. X_i \subset B \mbox{ für alle } i \in I 
  \mbox{ dann gilt } \cup_{i \in I}X\i \subset B 
  \mbox{ Also } \cup_{i\in I}X_I \mbox{ ist das Supremum } 
  \mbox{ für das Infinum: } 
  inf_{i \in I} X_i = \cap_{i \in I}X_i 
  \mbox{ es ist klar } \cap_{i \in I}X_i \leq X_i 
  \mbox{ Sei } C \subset M \mbox{ eine Menge mit } 
  C \leq X_i \forall i \in I, \mbox{ das bedeutet } 
  C < X_i \forall i \in I \mbox{ und insbesondere } 
  C \subset \cap_{i\in I}X_i \mbox{ also } C \leq \cap_{i \in I} X_i 
  \mbox{ und } \cap_{i \in I}X_i \mbox{ ist das Infinum }
\end{equation}
  \subsection{32}
\begin{equation}
  M ist eine endliche Menge 
  X_0 overset X_1 overset ... overset X_n overset ... 
  \mbox{ aufsteigende Fogle und } X_i \not = \emptyset \mbox{ für alle } i \in I 
  \mbox{ da M endlich ist } \exists x \in M \mbox{ und } k \in I \mbox{ mit } x \in X_n, \forall n \geq k 
  \mbox{ Eigentlich gibt es } k \in I \mbox{ mit }X_n = X_{k'} \mbox{ für alle }n\geq k' 
  \mbox{ Und wir wissen dass }inf_{i\inI}X_i = \cap_{i \in I}X_i = X_{k'} \not = \emptyset 
  \mbox{ weil die Folge absteigend ist und } [X_i \not \emptyset \forall i \in I ]
\end{equation}
  \subsection{33}
  \begin{equation}
  t = \tilde{s_2}x_1s_2s_0x_7 
  t' = \tilde{s_2}x_7s_2s_0x_1 
  t'' = \tilde{s_2}s_2x_1x_7s_0 
  \mbox{ Wir habenen die Fortsetzung einer Belegung }
  \alpha
  \alpha^{\wedge}(t) = \alpha^{\wedge}(t)(\tilde{s_2}x_1s_2s_0x_7)
  \mbox{ Sei }\alpha_i := \alpha(x_i)
  = \tilde{s_2}^{*}(\alpha(x_1)s_2^*(s_0^*\alpha(x_7)))
  = \alpha_1^{s_2^*(s_o^*\alpha_7)}= \alpha_1^{0 + \alpha_7} 
  \alpha(t') = \alpha_7^{0+\alpha1} 
  \alpha(t'') = (\alpha_1 + \alpha_7)^0 
  b) \alpha^{\wedge}(t) = \alpha_1^{\alpha_7} = 2^{17} = 13|072 
  \end{equation}

\end{document}
