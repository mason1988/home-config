
\documentclass[11pt]{amsart}
  \usepackage[utf8]{inputenc}
  \usepackage{amssymb,amsmath}
  \usepackage{verbatim}
  \usepackage{color}
  \usepackage{geometry}
  \usepackage{amsthm}
  \usepackage{stmaryrd}
  \usepackage{graphicx}
%-----------
\renewcommand{\baselinestretch}{2}

\newtheorem{defi}{Definition}
\newtheorem{axiom}{Axiom}
\newtheorem{nota}{Notation}
\newtheorem{prop}{Proposition}
\newtheorem{satz}{Satz}
\newtheorem{umf}{Umformung}

\newcommand{\words}{\Sigma^* \backslash \{\epsilon\}}
\newcommand{\etrans}[1]{\bar{\delta}(#1)}

\newenvironment{beweis}{\par\begingroup%
\settowidth{\leftskip}{\textsc{Beweis:~}}%
\noindent\llap{\textsc{Beweis:~}}}{\hfill$\Box$\par\endgroup}
%----------
\renewcommand{\P}{\mathbb{P}}
\renewcommand{\a}{\alpha}
\renewcommand{\,}{\quad}
\renewcommand{\o}{\omega}
\renewcommand{\O}{\Omega}
\newcommand{\s}{\sigma}
\renewcommand{\S}{\Sigma}
\renewcommand{\lim}[1]{\underset{#1}{lim}}
\newcommand{\m}[1]{\mbox{#1}}
\newcommand{\N}{\mathbb{N}}
\newcommand{\R}{\mathbb{R}}
\newcommand{\E}{\exists}
\newcommand{\A}{\forall}
\newcommand{\fr}[2]{\frac{#1}{#2}}
\newcommand{\ri}{\rightarrow}
\newcommand{\lori}{\longrightarrow}
\newcommand{\Ri}{\Rightarrow}
\newcommand{\Lri}{\Leftrightarrow}
\newcommand{\tri}{\mapsto} %tail right arrow
\newcommand{\ov}[2]{\overset{#1}{#2}}
\newcommand{\un}[2]{\underset{#1}{#2}}
\newcommand{\e}{\epsilon}
\newcommand{\busset}{\supset}
\newcommand{\inc}{\hspace*{0.5cm}}
\newcommand{\dinc}{\hspace*{1.2cm}}
\newcommand{\zz}{\overset{!}{=}}
\geometry{a4paper,left=2cm,right=2cm, top=1.5cm, bottom=1.5cm} 
\title{Zettel 09 }
\author{Florian Lerch(2404605)/Waldemar Hamm(2410010)}
\begin{document}
\maketitle

\subsection*{Aufgabe 1}

Es handelt sich hierbei um eine Hypergeometrische Verteilung. Gesucht ist der Schätzer, mit dem unsere Stichprobe die höchste Wahrscheinlichkeit aufweist.\\
Sei Y nun die Zufallsvariable, für die Anzahl der markierten Fische im 2. Fang. Als Erwartungswert ergibt sich: $E(Y) = X \fr{W}{N}$. \\
Wenn man die Hypergeometrische Verteilung betrachtet, so stellt man fest dass der auf eine ganze Zahl abgerundete Erwartungswert auch stets die höchste \\
Wahrscheinlichkeit aufweist. Abgerundet werden muss, weil es sich um eine diskrete Verteilung handelt und wir (hoffentlich) keine gebrochenen Fische aus dem \\
Wasser ziehen, so dass nur ganze Zahlen eine Wahrscheinlichkeit > 0 haben können. \\
Setzt man nun: $n = E(Y) \Ri n = X \fr{W}{N} \Ri N = X \fr{W}{n}$ wobei N gemäß der Aufgabenstellung die gesuchte Gesamtanzahl der Fische im Wasser ist, und \\
$\tau(\mathcal{X}) = [X \fr{W}{n}]$ unser maximum likelihood Schätzer ist.

\subsection*{Aufgabe 2}
\( s^2_n = \fr{1}{n-1}\sum_{i=1}^n(X_i - \bar{X})^2 = \fr{1}{n-1}\sum_{i=1}^n(X_i^2 - 2X_i\bar{X}^2 +\bar{X}^2) \) \\
\( = \fr{1}{n-1}\sum_{i=1}^n(X_i^2) - 2\bar{X}* \sum_{i=1}^n(X_i)  + \sum_{i=1}^n\bar{X}^2) \) \\
\( = \fr{1}{n-1}\sum_{i=1}^n(X_i^2) - 2\bar{X} * \sum_{i=1}^n(X_i) + n\bar{X}^2 \) \\
\( = \fr{1}{n-1}\sum_{i=1}^n(X_i^2) - 2\bar{X} *n\bar{X} + n\bar{X}^2 \) \\
\( = \fr{1}{n-1}\sum_{i=1}^n(X_i^2) - 2n\bar{X}^2 + n\bar{X}^2 \) \\
\( = \fr{1}{n-1}\sum_{i=1}^n(X_i^2) - n\bar{X}^2 \) \\
\( \Ri E(s^2_n) = \fr{1}{n-1}\sum_{i=1}^n(E(X_i^2)) - nE(\bar{X}^2)) \) \\
Sei nun allgemein $\sigma$ die Standardabweichung und $\mu$ der Erwartungswert, so dass gilt: \(E(X_i^2) = \sigma^2 + \mu^2 \mbox{ und } E(\bar{X}^2) = \fr{\sigma^2}{n} + \mu^2 \) \\
%\( = \fr{1}{n-1}( E(\sum_{i=1}^n(X_i^2)) - n E((\sum_{i=1}^n(X_i))^2) \) \\
\( = \fr{1}{n-1}(n(\sigma^2 + \mu^2) - n(\fr{\sigma^2}{n}+\mu^2)) = \fr{1}{n-1}(n\sigma^2-\sigma^2) = \sigma^2 = \) Varianz \\
%\( \fr{n\sigma^2 - \sigma^2}{n-1} ? \)
\newpage
\subsection*{Aufgabe 3}
\subsubsection*{a)}.\\
Für die gewöhnliche Binomialverteilung gilt: \\
\( P(X=k) = \binom{n}{k}p^kq^{n-k} \) für k Erfolge bei n Versuchen und Erfolgswahrscheinlichkeit p\\
Für die gewünschten Ereignisse muss es zunächst bis zum (r+k-1)'ten Versuch genau k Misserfolge und \\
r-1 Erfolge gegeben haben und anschließend muss der daurauf folgende (r+k)'te Durchlauf ein Erfolg sein \\
so dass wir dann genau auf r Erfolge und k Misserfolge kommen und alle Misserfolge bis dahin vor dem \\
(r+k)'ten Versuch liegen. \\
Setzt man nun in die Binomialverteilung ein und interpretiert die Misserfolge als gewünschtes \\
Ergebnis(also Erfolge in der Formel der Binomialverteilung) so erhält man:
 \( \binom{k+r-1}{k}p^{k+r-1 - k}q^k = \binom{k+r-1}{k}p^{r-1}q^k \) als Wahrscheinlichkeit \\
für den gewünschten Zustand bis zum (k+r-1)'ten Versuch und schließlich für den (k+r)'ten Versuch
aufgrund der Unabhängigkeit der Versuche (Bernoulli Experiment): \( f(k;r,p) = \binom{k+r-1}{k}p^{r-1}q^k * p = \binom{k+r-1}{k}p^rq^k \)\\ 
\subsubsection*{b)}.\\
Es handelt sich auch bei dieser Aufgabe um die negative Binomialverteilung: \( f(x;r,p) = \binom{x + r - 1}{x}p^{r}(1-p)^{x} \) \\
Gesucht ist der Maximum-Likelihood Schätzer für die Erfolgswahrscheinlichkeit (p) also $\bar{p}$ \\
Es gilt: \( ln(f(x;r,p)) = ln\binom{x+r-1}{x} + r*ln(p) + x(ln(1-p)) \) , wobei es sich also um die Logarithmusfunktion handelt, welche \\
am selben Punkt maximal wird, wie f(x;r,p) \\
Die erste Ableitung für p von dieser Funktion ist: \( ln(f(x;r,p))' = 0 + \fr{r}{p} - \fr{x}{1-p} = \fr{r}{p} - \f{x}{1-p} \) \\
Sei nun: \( \fr{r}{p} - \fr{x}{1-p} = 0 \Lri \fr{rp(1-p)}{p} - \fr{xp(1-p)}{1-p} = 0 \Lri r - rp - xp = 0 \Lri rp + xp = r \Lri p = \fr{r}{r+x} \) \\
$\Ri$ Der Maximum-Likelihood-Schätzer ist also $\bar{p} = \fr{r}{r+x}$ \\
Sei nun n = r+x, also die Anzahl der Durchläufe. Es gilt nun: \( E(\fr{r}{r+x}) = E(\fr{r}{n}) = \fr{1}{n}E(r) = \fr{1}{n}n*p = \fr{np}{n} = p \) \\
$\Ri$ Der Schätzer ist Erwartungstreu.

\subsection*{Aufgabe 4}
\subsubsection*{a)}.\\
Aus der Definion für fast sichere Konvergenz folgt: \\
\( \A \e > 0 : \E n_0: \A n \geq n_0: P(|X_n - X| \leq \e) = 1 \) \\
bzw.: \( \A \e > 0 : \E n_0: \A n > n_0: P(|X_n - X| < \e) = 1 \) \\
\( \Ri \A \e > 0 : \E n_0: \A n > n_0: P(|X_n - X| \geq \e) = 0 \) \\
\( \Ri \lim{n\ri \infty}P(|X_n - X| \geq \e) = 0 \) \\
\( \Ri X_n \ov{\P}{\lori} X \)
\subsubsection*{b)}.\\
\(X_n \ov{\P}{\lori} X \) \\
\( \Ri \lim{n\ri \infty}P(|X_n - X| \geq \e) = 0 \) \\
$\Ri$ Für genügend große n wird der Abstand zwischen $X_n$ und $X$ beliebig klein \\
$\Ri$ Für genügend große n wird der Abstand zwischen $P(X_n \leq x)$ und $P(X \leq x)$ beliebig klein \\ 
\( \Ri \A x:P(X \leq x) stetig \Ri \lim{n\ri \infty}P(X_n \leq x) = P(X \leq x) \) \\
\( \Ri X_n \ov{Law}{\lori} X \)

\end{document}
