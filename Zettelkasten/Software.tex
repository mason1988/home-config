% Created 2013-01-24 Do 01:28
\documentclass[11pt]{article}
\usepackage[utf8]{inputenc}
\usepackage[T1]{fontenc}
\usepackage{fixltx2e}
\usepackage{graphicx}
\usepackage{longtable}
\usepackage{float}
\usepackage{wrapfig}
\usepackage{soul}
\usepackage{textcomp}
\usepackage{marvosym}
\usepackage{wasysym}
\usepackage{latexsym}
\usepackage{amssymb}
\usepackage{hyperref}
\tolerance=1000
\providecommand{\alert}[1]{\textbf{#1}}

\title{Shortcutliste}
\author{florian}
\date{\today}
\hypersetup{
  pdfkeywords={},
  pdfsubject={},
  pdfcreator={Emacs Org-mode version 7.8.11}}

\begin{document}

\maketitle

\setcounter{tocdepth}{3}
\tableofcontents
\vspace*{1cm}
\section{einfache Tasten}
\label{sec-1}

\begin{description}
\item[AltGr - ü] Escape
\item[AltGr - ö] Tab
\item[AltGr - p] enter
\end{description}
\section{Org - Mode}
\label{sec-2}

\begin{description}
\item[C-c C-x C-i] Clock in
\item[C-c C-x C-o] Clock out
\item[C-c C-x C-c] cancel Clock
\item[C-c C-t] toggle todo state
\item[C-c C-x C-r] Zusammenfassung der Clocks von Substree
\item[C-u C-c C-x C-i] switch Task (nahtloser Wechsel)
\item[C-u C-u C-c C-x C-i] Clock in, mark as default (überall reclock in usw.)
\item[C-c C-x C-d] An jedem Tree jeweils die Zeit anzeigen
\item[C-c C-c] Update für Ding unter Cursor
\item[C-c C-x ;] Set (und start) Timer
\item[C-c C-x e] set estimated effort time
\item[C-c C-u] Sprung zum Header
\item[C-j/C-k] auf einer Ebene bewegen
\item[C-ö t] org tree to indirect buffer
\item[C-c C-x C-l] preview latex fragment (C-c C-c für undo)
\item[C-ö D] display inline images
\item[C-ä s] Screenshot erstellen und einsetzen
\item[C-ä l] konvertiere zu Latex
\item[C-ä o] org-babel-tangle
\item[C-ä .] org-capture
\end{description}
\section{Expansion/Completion}
\label{sec-3}

\begin{description}
\item[C-n,p] Evil Extension, zurück
\item[C-x C-n,p] Evil complete line, zurück
\item[M-ö] Hippie expand
\item[C-ä c] mögliche Completions anzeigen
\item[C-p] evil-paste-pop
        geht bei vorherigem paste alle
        elemente des kill-rings durch 
        (kill-ring speichert neue pos)
\item[C-ö ä] magpie expand (beginnt mit , und dann akronyme)
\item[C-ä ö] flosub, entweder aktuelles wort (ab leer oder \n) oder markierung
\item[C-ä b] flosub mit start und end durch leerzeichen getrennt
\end{description}
\section{sprünge}
\label{sec-4}

\begin{description}
\item[g ;] springe zu letzter veränderung
\item[C-ö n/r] next/prev Empty Line
\item[C-ä i/u] next/prev regex
\item[C-ö g/G] Vertikalsprung hoch/runter
\item[M-ä] Cursor in Fenster springen lassen
\item[C-x p ret] setze/entferne autonamedbookmark
\item[C-x j n] cycle bookmarks current file
\item[C-M-n] jump up
\end{description}
\section{Snippets}
\label{sec-5}

\begin{description}
\item[C-ä n] Neues Snippet
\item[C-c C-c] Snippet Buffer laden
\item[C-ä f] neues Snippet aus Content
\item[C-ä g] Platzhalter erstellen (für Oneshot Snippet)
\item[C-ä h] Oneshot Snippet (erstellen oder einsetzen)
\end{description}
\section{sonstiges}
\label{sec-6}

\begin{description}
\item[C-x * c] öffnet calculator
                 => eingabe in postfix/ergebnis paste mit y/close mit q
\item[C-ö d] doc-view-mode
\item[C-ä k/K] encrypt/decrypt region
\item[S-, S-''] minimize(/restore) client
\item[S-.] letzter Tag
\item[S-Shift-i] restore client
\item[C-x s] save-some-buffers
\item[M-w] buffer back
\item[C-ä j] evil normal state
\item[C-ä v] revert buffer
\item[C-ü C-q] toggle read only
\item[S-m] startet Maus modus
\item[q] start/end dragging
  iae      :: taste 1,2,3
  b        :: lasse maus springen
  u        :: undo
\end{description}
\section{Fenstermanagement}
\label{sec-7}

\begin{description}
\item[C-ä w s] Session speichern
\item[C-ä w r] Session laden
\item[C-ä w u] Winner undo
\item[C-g / C-t 0] Popwin Fenster schließen
\item[C-t] Popwin Keymap:
\item[b] Popup Buffer
\item[l] Popup Last Buffer
\item[s] stick Popup Window
\item[spc] select Popup Window
\item[e] show messages
\item[C-t C-u \ldots{}] zwingt das Fenster zum öffnen in popwin
\item[C-ü ö b] view Buffer other window
\item[C-ü ö f] find file other window
\item[C-ö b] display buffer
\item[C-ö f] display file
\item[C-ä w d/D] dedicate/undedicate window
\item[C-ä v] revert buffer
\end{description}
\section{helm}
\label{sec-8}

\begin{description}
\item[M-m] Helm: M-a = alle markieren
\item[C-ö o] Helm Occur (akt. Buffer)
\item[M-h M-x] Helm - M-x -> History usw.
\item[C-ö s] Helm - do - grep
\item[C-ö k] helm show killring
\item[C-ö h] helm apropos
\item[C-ö i] Imenu (Header Übersicht und Sprung)
\item[C-ö a] org-headlines (komplette übersicht)
\end{description}
\section{emms}
\label{sec-9}

\begin{description}
\item[C-ö e n] emms-next
\item[C-ö e p] emms-previous
\item[C-ö e P] emms-pause
\item[C-ö e s] emms-shuffle
\item[C-ö e r] emms-repeat
\item[C-ö e f] emms-add-find
\item[C-ö e d] emms-add-directory-tree
\item[C-ö e l] emms-playlist-mode-go
\end{description}
\section{repeat}
\label{sec-10}

\begin{description}
\item[C-x z z\ldots{}] repeat last emacs action
\item[C-!] evil-normale-state (force)
\end{description}
. @@          :: repeat last macro
\begin{description}
\item[C-x r] repeat
\item[C-ä r] repeat
\end{description}
\section{magit}
\label{sec-11}

\begin{description}
\item[C-ö m l] magit pull
\item[C-ö m h] magit push
\item[C-ö m s] magit status
    s         :: stage
    u         :: unstage
    c C-c C-c :: commit, - absenden
    ll        :: log
\end{description}
                 
\section{Firefox}
\label{sec-12}

\begin{description}
\item[;y] kopiere linkurl
\item[``+;y] kopiere linkurl in clipboard
\item[A] toggle cur bookmark
\item[:bmarks!] bookmarks durchsuchen
\end{description}

\end{document}