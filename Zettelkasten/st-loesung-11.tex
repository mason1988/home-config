
\documentclass[11pt]{amsart}
  \usepackage[utf8]{inputenc}
  \usepackage{amssymb,amsmath}
  \usepackage{verbatim}
  \usepackage{color}
  \usepackage{geometry}
  \usepackage{amsthm}
  \usepackage{stmaryrd}
  \usepackage{graphicx}
%-----------
\renewcommand{\baselinestretch}{2}

\newtheorem{defi}{Definition}
\newtheorem{axiom}{Axiom}
\newtheorem{nota}{Notation}
\newtheorem{prop}{Proposition}
\newtheorem{satz}{Satz}
\newtheorem{umf}{Umformung}

\newcommand{\words}{\Sigma^* \backslash \{\epsilon\}}
\newcommand{\etrans}[1]{\bar{\delta}(#1)}

\newenvironment{beweis}{\par\begingroup%
\settowidth{\leftskip}{\textsc{Beweis:~}}%
\noindent\llap{\textsc{Beweis:~}}}{\hfill$\Box$\par\endgroup}
%----------
\renewcommand{\P}{\mathbb{P}}
\renewcommand{\a}{\alpha}
\renewcommand{\,}{\quad}
\newcommand{\m}[1]{\mbox{#1}}
\newcommand{\N}{\mathbb{N}}
\newcommand{\R}{\mathbb{R}}
\newcommand{\E}{\exists}
\newcommand{\A}{\forall}
\newcommand{\fr}[2]{\frac{#1}{#2}}
\newcommand{\ri}{\rightarrow}
\newcommand{\Ri}{\Rightarrow}
\newcommand{\Lri}{\Leftrightarrow}
\newcommand{\tri}{\mapsto} %tail right arrow
\newcommand{\ov}[2]{\overset{#1}{#2}}
\newcommand{\un}[2]{\underset{#1}{#2}}
\newcommand{\busset}{\supset}
\newcommand{\inc}{\hspace*{0.5cm}}
\newcommand{\dinc}{\hspace*{1.2cm}}
\newcommand{\zz}{\overset{!}{=}}
\geometry{a4paper,left=2cm,right=2cm, top=1.5cm, bottom=1.5cm} 
\title{Zettel 11 }
\author{Florian Lerch(2404605)/Waldemar Hamm(2410010)}
\begin{document}
\maketitle

\subsection*{Aufgabe 1}
Statistisches Modell: \\
Das statistische Modell sei $(\mc{X}, \mc{F}, P_{\theta} : \theta \in \Theta)$ mit: \\ 
Sei Y = Ausgang des Experiments(=001101...), $\mc{X} = [0,10]\subset \N \m{mit} \mc{X} \ni x = \Pi_{y \in Y}y$ \\
$\mc{F} = \mathfrak{P}(\mc{X})$ also die Potenzmenge von $\mc{X}$ \\
$\Theta = [0,1]$ \\
$P_{\theta} =$ Binomialverteilung mit den Parametern $n=10$ und $p = \theta$ \\ % muss die Verteilung die Stichprobe so wie sie ist als Input nehmen?
Likelihood Funktion: \\
$L(\theta | x) = P_{\theta}(x)$ \\ 
$= \binom{10}{x} \theta^{x} * (1-\theta)^{10-x}$ \\
Maximum likelihood Schätzer: \\
Da die logarithmus Funktion von $L(\theta)$ an der selben Stelle ihr maximum erreicht, reicht es, diese zu maximieren. \\
alternativ, falls x = Anzahl: $= log(L(\theta))  = log(\binom{10}{x}) + x*log(\theta) + (10-x)*log(1 - \theta)$ \\
$\Ri \fr{d}{d\theta}log(L(\theta)) = \fr{x}{\theta} - \fr{10-x}{1 - \theta}$ \\ 
Sei nun $\fr{x}{\theta} - \fr{10-x}{1-\theta} = 0$ \\
$\Ri \fr{x-x\theta}{\theta - \theta^2} - \fr{10\theta - x\theta}{\theta - \theta^2} = 0$ \\
$\Ri \fr{4-4\theta}{\theta - \theta^2} - \fr{10\theta - 4\theta}{\theta - \theta^2} = 0$ \\
$\Ri \fr{4-10\theta}{\theta - \theta^2} = 0$ \\
$\Ri 4-10\theta = 0$ \\
$\Ri 4 = 10\theta$ \\
$\Ri \theta = 0,4$ = Maximum likelihood Schätzer\\

\subsection*{Aufgabe 2}
\subsubsection*{a)} Statistisches Modell: \\
Seien $x_1,x_2,...,x_n$ die Poisson-verteilten Werte \\
Das statistische Modell ist dann $(\mc{X},\mc{F},P_{\nu}:\nu \in \Theta)$ mit \\
% \nu entspricht wohl lambda, und ist daher die ereignisanzahl in einem entsprechenden intervall
$\mc{X} = \R_+^n$ \\
$\mc{F} = \mathfrak{P}(\R_+^n) = \mathfrak{P}(\mc{X})$ \\
$\Theta = \N_+$ \\
$P_{\nu}$ = Poissonverteilung zum Parameter $\nu$ \\
\subsubsection*{b)} 
Maximum-Likelihood-Schätzer: \\
Nach Def. der Poissonverteilung gilt: $P_{\nu}(X=x) = \fr{\nu^x}{x!}exp(-\nu)$ \\
% also Ws dafür, dass im gewünschten intervall x positive ergebnisse gemessen werden
$\Ri L(\nu) = \fr{\nu^{x_1 + x_2 + ... + x_n}}{x_1! * x_2! * ... * x_n!}exp(-n\nu)$ \\
$\Ri log(L(\nu)) = (x_1 + x_2 + ... + x_n)log(\nu) - log(x_1! * x_2! * ... * x_n!) - n\nu$ \\
$\Ri \fr{d}{d\nu}log(L(\nu)) = \fr{x_1 + x_2 + ... + x_n}{\nu} - 0 - n$ \\
Sei nun $log(L(\nu)) = 0 \Lri \fr{x_1 + x_2 + ... + x_n}{\nu} - n = 0$ \\
$\Ri (\sum_{i=1}^n x_i) - \nu n = 0$ \\
$\Ri (\sum_{i=1}^n \fr{x_i}{n}) = \nu$ \\
$\Ri T_{ML} = (\sum_{i=1}^n \fr{X_i}{n})$ \\
\subsubsection*{c)}
Erwartungstreue: \\
$E(T_{ML}) = E(\sum_{i=1}^n \fr{X_i}{n}) = \fr{1}{n} E(\sum_{i=1}^n X_i) = \fr{1}{n} \sum_{i=1}^n E(X_i) = \fr{n}{n} \mu = \mu$ \\
$\Ri$ Da der Erwartungswert bei der Poissonverteilung gleich dem Parameter (in diesem Fall $\nu$) ist, ist $T_{ML}$ erwartungstreu \\
\subsubsection*{d)}
Varianz: \\
$V_{\nu}[T_{ML}] = V_{\nu}(\sum_{i=1}^n \fr{X_i}{n}) = \sum_{i=1}^nVar(\fr{1}{n}X_i) = \sum_{i=1}^n(\fr{1}{n^2}Var(X_i))$ \\
$= \fr{1}{n^2}\sum_{i=1}^n\sigma^2 = \fr{n\sigma^2}{n^2} = \fr{\sigma^2}{n} = \fr{\mu}{n} = \fr{\nu}{n}$ \\
\subsubsection*{e)}
Fisher-Information: \\
$I(\nu) = -E[(\fr{d^2}{d\nu^2}log(f_{\nu}(X)))]$ \\
$= -E[(\fr{d^2}{d\nu^2} xlog(\nu)-log(x!)-\nu)]$ \\
$= -E[(\fr{d}{d\nu} \fr{x}{\nu} - 0 - 1)]$ \\
$= -E[\fr{-x}{\nu^2})]$ \\
$= -(\fr{-\nu}{\nu^2}) = -(-\fr{1}{\nu}) = \fr{1}{\nu} = \fr{1}{\mu} = \fr{1}{\sigma^2}$ \\
\subsubsection*{f)}
Cramer Rao Schranke:
$Var(T_ML) = \fr{\nu}{n}$ \\
Schranke = $\fr{1}{I(\nu)} = \nu$ \\
$\Ri$ Der Schätzer liegt auf der Schranke.

\subsection*{Aufgabe 3}
Seien $x_u$ und $x_o$ die untere und obere Grenze des Konfidenzintervalls: \\
$\Ri P(x_u \leq b \leq x_o) = 1 - \alpha$ \\
$\Ri F(x_o) - F(x_u) = 1 - \alpha$ \\
Sei $x_m = max(X_1,...,X_n)$ \\
$\Ri x_u = x_m$ da b nicht kleiner als $x_m$ sein kann. \\
Weiter weiß ich auch nicht, stimmt denn der Anfang so?

\subsection*{Aufgabe 4}
-
