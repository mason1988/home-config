
\documentclass[11pt]{amsart}
  \usepackage[utf8]{inputenc}
  \usepackage{amssymb,amsmath}
  \usepackage{verbatim}
  \usepackage{color}
  \usepackage{geometry}
  \usepackage{amsthm}
  \usepackage{stmaryrd}
  \usepackage{graphicx}
%-----------
\renewcommand{\baselinestretch}{2}

\newtheorem{defi}{Definition}
\newtheorem{axiom}{Axiom}
\newtheorem{nota}{Notation}
\newtheorem{prop}{Proposition}
\newtheorem{satz}{Satz}
\newtheorem{umf}{Umformung}

\newcommand{\words}{\Sigma^* \backslash \{\epsilon\}}
\newcommand{\etrans}[1]{\bar{\delta}(#1)}

\newenvironment{beweis}{\par\begingroup%
\settowidth{\leftskip}{\textsc{Beweis:~}}%
\noindent\llap{\textsc{Beweis:~}}}{\hfill$\Box$\par\endgroup}
%----------
\renewcommand{\P}{\mathbb{P}}
\renewcommand{\a}{\alpha}
\renewcommand{\,}{\quad}
\newcommand{\m}[1]{\mbox{#1}}
\newcommand{\N}{\mathbb{N}}
\newcommand{\R}{\mathbb{R}}
\newcommand{\E}{\exists}
\newcommand{\A}{\forall}
\newcommand{\fr}[2]{\frac{#1}{#2}}
\newcommand{\ri}{\rightarrow}
\newcommand{\Ri}{\Rightarrow}
\newcommand{\Lri}{\Leftrightarrow}
\newcommand{\tri}{\mapsto} %tail right arrow
\newcommand{\ov}[2]{\overset{#1}{#2}}
\newcommand{\un}[2]{\underset{#1}{#2}}
\newcommand{\busset}{\supset}
\newcommand{\inc}{\hspace*{0.5cm}}
\newcommand{\dinc}{\hspace*{1.2cm}}
\newcommand{\zz}{\overset{!}{=}}
\renewcommand{\baselinestretch}{2}
\geometry{a4paper,left=2cm,right=2cm, top=1.5cm, bottom=1.5cm} 

\title{Zettel 11}
\author{Florian Lerch(2404605)/Waldemar Hamm(2410010)}

\begin{document}
\maketitle

\subsection*{Aufgabe 34}
\subsubsection*{a)}% Induktion sortieren <<FIXME th11.1>>
Induktionsanfang(y=1): \\ %keine Ahnung
\(\dinc ack(1,1) = ack(0+1,0+1) = ack(0,ack(0+1,0)) = ack(0,ack(1,0)) = ack(0,1) = 2 \)\\
\(\dinc 2*0 = 2 \)\\
Induktionsvorraussetzung: \\ %glaube nicht dass das die IV ist (formal)
\(\dinc ack(1,y) = 2 * y \m{gilt für ein y} \in \N \geq 1 \) \\
Induktionsschritt: \\
z.Z.: $ack(1,y+1) = 2*(y+1) = 2y + 2 $ \\
$ack(1,y+1) = ack(0,ack(1,y)) =_{I.V} ack(0,2*y) = 2y+2$
\subsubsection*{b)}%FIXME wieder das gleiche
Induktionsanfang(y=0): \\
\(\dinc ack(2,0) = ack(1+1,0) = 1 \) \\
\(\dinc 2^0 = 1 \) \\
Induktionsvorraussetzung(y=0): \\
\( ack(2,y) = 2^y \m{ gilt für ein } y \geq 0  \in \N \) \\
Induktionsschritt: \\
z.Z.: \( ack(2,y+1) \zz 2^{y+1} = 2^{y}*2 \) \\
\( ack(2,y+1) = ack(1,ack(2,y)) =_{I.V.} ack(1,2^{y}) = 2^{y}*2\)

\subsection*{Aufgabe 35} % <<FIXME th11.2>>
\(h: \mathbb{N}^2 \rightarrow \mathbb{N}\) \\
Sei \(h(a,b) := \begin{cases} a-b = sub(a,b) &\mbox{(falls } \mbox{a gerade)} \\ \mbox{nicht definiert} &\mbox{(sonst)} \end{cases}\)

\subsection*{Aufgabe 36}
% was wird alles zu alpha, bei loop wars das komplette teil
% theoretisch könnte man noch in unterpunkte splitten
\( \a_1 := X_2 := X_2 - 1 \) \\
\( \a_2 := \underline{while} X_2 \not = 0 \underline{do} X_2 := X_2 - 1 ; X_1 := X_1 - 1 \underline{od} \) \\

\( [[\a_1]]^{(2)}(\a_1,\a_2) = [[\a_1 := X_2 := X_2 - 1]]^{(2)}(\a_1,\a_2) \) \\
\( \dinc = (\a_1,\a_2 - 1) \)

\( [[\a_2]]^{(2)}(\a_1,\a_2) = [[\underline{while} X_2 \not = 0 \underline{do} X_2 := X_2 - 1 ; X_1 := X_1 - 1 \underline{od}]]^{(2)}(\a_1,\a_2) \) \\
\( \dinc = (\a_1 - 1, \a_2 - 1)^{\a_2} \) \\ 

% out(projektion) * schleifen * schleifen-1 * ... * schleife1 * in
\( [[P]](\a_1) = (\underline{out}^{(2)}_1 \circ [[\a_2]]^{(2)} \circ [[\a_1]]^{(2)} \circ \underline{in}^{(2)}_2)(\a_1,\a_2) \) \\
\( \dinc = (\underline{out}^{(2)}_1 \circ [[\a_2]]^{(2)} \circ [[\a_1]]^{(2)})(\a_1,\a_2) \) \\
\( \dinc = (\underline{out}^{(2)}_1 \circ [[\a_2]]^{(2)})(\a_1,\a_2 - 1) \) \\
\( \dinc = (\underline{out}^{(2)}_1) (\a_1 - (\a_2-1), 0) \) \\
\( \dinc = \a_1 - (\a_2 - 1) \) \\
\( \Rightarrow [[P_n]] : \N^2 \ri \N , (\a_1,\a_2) \tri (\a_1 - (\a_2 -1)) \) \\

\end{document}
