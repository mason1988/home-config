
\documentclass[11pt]{amsart}
\usepackage[utf8]{inputenc}
\usepackage{amssymb,amsmath}
\usepackage{verbatim}
\usepackage{color}
\usepackage{geometry}
\geometry{a4paper,left=2cm,right=2cm, top=1.5cm, bottom=1.5cm} 
\usepackage{amsthm}
\usepackage{stmaryrd}
\usepackage{graphicx}

%\includegraphics{?} setzt bild ein
%\ref{labelname} erstellt link zu labelname
%\label{labelname} kann einfach irgendwo drangesetz werden

\newtheorem{defi}{Definition}
\newtheorem{axiom}{Axiom}
\newtheorem{nota}{Notation}
\newtheorem{prop}{Proposition}
\newtheorem{satz}{Satz}
\newtheorem{umf}{Umformung}

\newenvironment{beweis}{\par\begingroup%
\settowidth{\leftskip}{\textsc{Beweis:~}}%
\noindent\llap{\textsc{Beweis:~}}}{\hfill$\Box$\par\endgroup}

\renewcommand{\baselinestretch}{1}
\newcommand{\words}{\Sigma^* \backslash \{\epsilon\}}
\newcommand{\etrans}[1]{\bar{\delta}(#1)}
\renewcommand{\P}{\mathbb{P}}

\title{Zettel 8}
\author{Florian Lerch(2404605)/Waldemar Hamm(2410010)}
%\date{} % Activate to display a given date or no date (if empty),
% otherwise the current date is printed 

\begin{document}
\maketitle

\subsection*{Aufgabe 1}

\subsubsection*{a)}

Der Raum $\Omega$ soll die Anzahl der Buben enthalten die ein Spieler jeweils in der Hand hält. Da es nur 4
Buben gibt, gilt also: $\Omega = \{0,1,2,3,4\}$. $\mathbb{P}: \Omega \rightarrow [0,1]$ soll nun also  die Wahrscheinlichkeit
dafür darstellen, dass ein Spieler die jeweilige Anzahl Buben in seinen 10 Karten besitzt.
Bei 32 Karten und 4 Buben liegt die Wahrscheinlichkeit bei jeder einzelnen zugeteilten Karte bei $\frac{4}{32} =
\frac{1}{8}$ dafür, dass es sich um einen Buben handelt.\\
Da die Karten alle direkt zugeteilt werden und wir nur die Wahrscheinlichkeit für alle 10 Karten zusammen betrachten,
beeinflussen sich die einzelnen Karten in ihrer Wahrscheinlichkeit nicht wir können somit die Binomialverteilung
für $\mathbb{P}$ verwenden.\\
Es ergibt sich somit: $\mathbb{P}(\omega) = \binom{10}{\omega}*(\frac{1}{8})^{\omega}*(\frac{7}{8})^{10-\omega}$ für $\omega \in \Omega$

\subsubsection*{b)}

Aus Sicht des jeweiligen Spielers befinden sich nun noch 4 - X Karten im Spiel. Für die Karten im Skat gilt
daher das selbe Prinzip wie schon in a), d.h. Binomialverteilung. \\
Sei $\Omega' = \{0,1,2\}$ und somit also die möglichen Anzahlen an Buben im Skat. \\
Analog zu a) ergibt sich für $\mathbb{P}(Y|X = k)$ nun für 2 Kartenziehungen und einer Wahrscheinlichkeit
von $\frac{4-X}{32}$ für einen Buben pro Karte:\\
Für $\omega \in \Omega:$ $\mathbb{P}(Y = \omega |X = k) = \binom{32}{\omega}(\frac{4-X}{32})^{\omega} * (1 - \frac{4-X}{32})^{2 - \omega}$

\subsection*{Aufgabe 2}

\subsubsection*{a)}
Ohne Betrachtung von X gilt zunächst: $Y$ bildet auf $[2,12] \subset N$ \\
Ferner biledet X auf $[1,6] \subset N$ ab, mit gleichen Wahrscheinlichkeiten der Werte, es gilt also: $P(X=x) = \frac{1}{6}$ für $x \in [1,6]$ \\
$\Rightarrow P(Y = y | X = k) = \frac{P(X=k , Y = y)}{P(X = k)} = \frac{P(X=k , Y = y)}{6}$ \\
$ = \begin{cases} \frac{1}{6} &\mbox{falls } k < y \leq 6+k \\ 0 &\mbox{sonst} \end{cases}$ \\

\subsubsection*{b)}
$g(k) := E[Y|X=k] = \sum_yy*P(Y=y | X = k) = \sum_{k < y \leq k+6}y*\frac{1}{6} = \frac{1}{6} * (k+1 + ... + k+6) = \frac{21}{6}*k = 3,5k$

\subsubsection*{c)}
Sei Z die Augenzahl des 2. Wurfes, so das gilt Y = X+Z \\
$\Rightarrow E[Y] = E[X+Z] = E[X]+E[Z] = 3,5 + 3,5 = 7$ \\
$E[g(X)] = E[E(Y|X)] = E[\sum_yy*P(Y=y | X )] = \sum_x[\sum_yy*P(Y=y|X=x)]*P(X=x)$ \\
$= \sum_x\sum_yy*P(Y=y|X=x)*P(X=x) = \sum_yy*\sum_xP(Y=y, X=x) = \sum_yy*P(Y=y) = E(Y) = 7$ \\

\subsection*{Aufgabe 3}

\subsubsection*{a)}

Da X und Y gleichverteilt sind, gilt: $Var(X) = Var(Y) \rightarrow Var(X) - Var(Y) = 0$\\
Durch die symmetrie der Kovarianz lässt sich umformen:\\
$Cov(X+Y, X-Y) = Cov(X,X-Y) + Cov(Y,X-Y) = Cov(X-Y,X) + Cov(X-Y, Y) = Cov(X,X) - Cov(Y,X) + Cov(X,Y) - Cov(Y,Y)$\\
$ = Cov(X,X) - Cov(Y,Y) = Var(X) - Var(Y) = 0$

\subsubsection*{b)}

Für Unabhängigkeit müsste gelten: $\mathbb{P}([X+Y]*[X-Y]) = \mathbb{P}(X+Y)*\mathbb{P}(X-Y) \Leftrightarrow \mathbb{P}(X^2 - Y^2) = \mathbb{P}(X+Y)*\mathbb{P}(X-Y)$ \\
Es gelte $\mathbb{P}(z) = \begin{cases} 1 &\mbox{falls } z=-1 \\ 0 &\mbox{sonst} \end{cases}$
\begin{tabbing}
Sei X = 0 und Y = 1 \=$\Rightarrow \mathbb{P}(X^2-Y^2) = \mathbb{P}(-1) = 1$ \\
\> $\Rightarrow \mathbb{P}(X+Y)*\mathbb{P}(X-Y) = \mathbb{P}(1)*\mathbb{P}(-1) = 0*1 = 0 \not = 1$

\end{tabbing}
$\Rightarrow$ in diesem Beispiel sind die Zufallsvariablen X+Y und X-Y zwar unkorelliert (Kovarianz ist 0) aber nicht unabhängig.

\subsection*{Aufgabe 4}

\subsubsection*{a)}

Die Wahrscheinlichkeit für einen erfolgreichen Wurf (eine 1) liegt bei $\frac{1}{5}$ und für einen 
nicht erfolgreichen Wurf (ungleich 1) somit bei $1 - \frac{1}{5} = \frac{4}{5}$ \\
Da die einzelnen Würfe keinen Einfluss aufeinander nehmen und jeder Wurf klar in Erfolg und Misserfolg 
getrennt werden kann, lässt sich die Varianz der Normalverteilung verwenden, und es ergibt sich: \\
$Var(S_n) = n * \frac{1}{5} * \frac{4}{5} =  \frac{4n}{25}$ \\
$\Rightarrow Var(\frac{S_n}{n}) = \frac{4}{25n}$ \\
Für den Erwartungswert gilt aufgrund der Binomialverteilung: $E(S_n) = \frac{n}{5}$ \\
$\Rightarrow E(\frac{S_n}{n}) = \frac{1}{5}$ \\
Eingesetzt in die Ungleichung ergibt sich somit: $P[|\frac{S_n}{n} - \frac{1}{5}| < \epsilon] \geq 1 - \frac{4}{25n * \epsilon^2}$

\subsubsection*{b)}

Es soll gelten: $1 - \frac{4}{25n * 0.000001} > 0.95$ \\
$\Leftrightarrow 1-0.95 > \frac{4}{25n * 0.000001}$ \\
$\Leftrightarrow 0.05 > \frac{160000}{n}$ \\
$\Leftrightarrow n > 3 200 000$

\subsection*{Aufgabe 5}

\subsubsection*{a)}

$G_j = \{ (j,\omega_2,\omega_3) | \omega_2 \in \{ 1,2,3 \}, \omega_3 \in \{ 1,2,3 \} \backslash  \{ j , \omega_2 \} \}$ \\
        $= \{ \omega \in \Omega | \omega_1 = j \wedge \omega_3 \not = j \wedge \omega_3 \not = \omega_2\ \wedge \omega_3 \not = j \}$ \\
$W_k = \{ ( \omega_1 , k , \omega_3 ) | \omega_1 \in \{ 1,2,3 \} , \omega_3 \in \{ 1, 2, 3 \} \backslash \{\omega_1 , k \} \}$ \\
     $= \{ \omega \in \Omega | \omega_2 = k \wedge \omega_3 \not = k \wedge \omega_3 \not = \omega_1 \wedge \omega_3 \not = k \}$ \\
$M_l = \{ ( \omega_1 , \omega_2 , l ) | \omega_1 \in \{ 1,2,3 \} \backslash \{ l \} , \omega_2 \in \{ 1, 2, 3 \}  \backslash \{ l \} , l \}$ \\
     $= \{ \omega \in \Omega | \omega_1 \not = l \wedge \omega_2 \not = l \wedge \omega_3 = l \}$ \\

$P(G_j | W_k \cap M_l, 1 \leq j,k,l \leq 3) = \frac{P( M_l | W_k , G_j) P(G_j | W_k)}{P(M_l | W_k)}$ \\
$P(M_l | W_k , G_j ) = 1$ , für $l \not = k$ und $l \not = j$, was immer der Fall ist
$P(G_j | W_k) = \frac{1}{3}$ , da es keine Beeinflussung durch $W_k$ gibt
$P(M_l | W_k) = \frac{1}{2}$ , da für l nur noch 2 Werte bleiben
$P(G_j | W_k , M_l) = \frac{1 * \frac{1}{3}}{\frac{1}{2}} = \frac{2}{3}$

Der Spieler sollte die Tür also auf jeden Fall wechseln, da die Wahrscheinlichkeit, dass der Gewinn
hinter der anderen Tür liegt, bei 2/3 liegt, wohingegen, die Wahrscheinlichkeit der jetztigen Tür
nur bei 1/3 liegt.

\subsubsection*{b)}



\subsubsection*{c)}

$\Omega = \{(1,2),(1,3),(2,3),(3,2)\}$ \\
Für den Spieler gibt es beim ersten Schritt also 3 Möglichkeiten: Tor 1, 2 oder 3. \\ 
Falls der Spieler Tor 2 oder 3 wählt, so würde er beim wechsel auf der richtigen Tür landen und gewinnen. \\
Nur bei der Wahl von Tor 1 würde er verlieren, so dass sich als Erfolgswahrscheinlichkeit $\frac{2}{3}$ ergibt. \\
Analog dazu liegt die Erfolgswahrscheinlichkeit bei der "nie wechseln Strategie" nur bei $\frac{1}{3}$.

\subsection*{Aufgabe 6}

Sei A das Ereigniss einer gleichen Dna und B eines positives Tests, sowie A' und B' jeweils das Gegenteil.\\
Es gilt: $P(A) = \frac{1}{10^6} \Rightarrow \frac{A'} = \frac{999999}{10^6}$ \\
$P(B | A) = 1$ und $P(B | A') = \frac{1}{10^5}$ \\
$P(B) = P(B \cap A) + P(B \cap A') = P(B|A) * P(A) + P(B|A') * P(A')$ \\
$= \frac{1}{10^6} + \frac{000000}{10^{11}} = \frac{100000 + 999999}{10^{11}}$ \\
$\frac{1099999}{10^{11}} \approx \frac{11}{10^6}$ \\ 
$\Rightarrow P(A | B) = \frac{P(B|A) * P(A)}{P(B)} = \frac{1 * \frac{1}{10^6}}{\frac{10}{10^6}} = \frac{1}{11}$ \\
Die Wahrscheinlichkeit dafür, dass das DNA Profil eines zufällig getestetes Menschen, mit positivem Testergebniss, 
tatsächlich mit dem DNA Profil der Probe vom Tatort übereinstimmt, liegt also bei grade mal $\frac{1}{11}$ \\

\end{document}
