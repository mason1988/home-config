
\documentclass[11pt]{amsart}
\usepackage[utf8]{inputenc}
\usepackage{amssymb,amsmath}
\usepackage{verbatim}
\usepackage{color}
\usepackage{geometry}
\geometry{a4paper,left=2cm,right=2cm, top=1.5cm, bottom=1.5cm} 
\usepackage{amsthm}
\usepackage{stmaryrd}
\usepackage{graphicx}

%\dincludegraphics{?} setzt bild ein
%\ref{labelname} erstellt link zu labelname
%\label{labelname} kann einfach irgendwo drangesetz werden

\newtheorem{defi}{Definition}
\newtheorem{axiom}{Axiom}
\newtheorem{nota}{Notation}
\newtheorem{prop}{Proposition}
\newtheorem{satz}{Satz}
\newtheorem{umf}{Umformung}

\newenvironment{beweis}{\par\begingroup%
\settowidth{\leftskip}{\textsc{Beweis:~}}%
\noindent\llap{\textsc{Beweis:~}}}{\hfill$\Box$\par\endgroup}

\renewcommand{\baselinestretch}{1}
\newcommand{\words}{\Sigma^* \backslash \{\epsilon\}}
\newcommand{\etrans}[1]{\bar{\delta}(#1)}
\renewcommand{\P}{\mathbb{P}}

\title{Zettel 10}
\author{Florian Lerch(2404605)/Waldemar Hamm(2410010)}
%\date{} % Activate to display a given date or no date (if empty),
% otherwise the current date is printed 

\begin{document}
\maketitle

\subsection*{Aufgabe 35}

\subsubsection*{a)}
\( \alpha^{\v}(t) = \{1,2,3,4,5,6,7\} \backslash ( \{1\} \backslash \{1,2,3,4,5\} \) \\
\( = \{0,1,2,3,4,5,6,7\} \backslash \emptyset = \{1,2,3,4,5,6,7\} \)

\subsubsection*{b)}
\( \alpha^{\w}(P) \mbox{ für } p_1t \mbox{ ist } 1\) bzw. wahr da \( | \{0,1,2,3,4,5,6,7\} | \leq 100 \) \\
\( \alpha^{\w}(P) \mbox{ für } \tilde{p}_1t \mbox{ ist } 1\) bzw. wahr da \( | \{0,1,2,3,4,5,6,7\} | = 8\) und 8 ist gerade \\

\subsubsection*{c)}
\( \alpha^{\w}(A) \mbox{ für } A = \bigw_{x_5}p_1t \mbox{ ist wahr} \)

\subsubsection*{d)}
\( \alpha^{\w}(B) \mbox{ für } B = (\bigw_{x_5}p_1t)\w(\bigv_{x_4}\tilde{p}_1t) \) ist 1 bzw. wahr.

\subsection*{Aufgabe 36}

\subsubsection*{a)}
\( \alpha^{\w}(t) =  5*4^2*3^2 = 720 \)

\subsubsection*{b)}
\( a^{\w}(P) = 1 \mbox{ bzw. wahr da } (2 * 5) < (5 * 4^2 * 3^2) \Leftrightarrow 10 < 720 \)

\subsubsection*{c)}
Sei \( x_3 = 0 \Rightarrow \alpha^{\v}(t) = 0 \Rightarrow\mbox{ es existiert kein } m \in \mathbb{N} 
\mbox{ so dass gilt: } (2m < 0) \Rightarrow \alpha^{\v}(A) = 0  \)

\subsubsection*{d)}
Es gilt die selbe Begründung wie schon in c), also $\alpha^{\v}(B) = 0$, da es in jedem Fall
ein $x_3 \in \mathbb{N}$ gibt, so dass die Aussage falsch ist.

\end{document}
