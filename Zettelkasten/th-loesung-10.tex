
\documentclass[11pt]{amsart}
\usepackage[utf8]{inputenc}
\usepackage{amssymb,amsmath}
\usepackage{verbatim}
\usepackage{color}
\usepackage{geometry}
\geometry{a4paper,left=2cm,right=2cm, top=1.5cm, bottom=1.5cm} 
\usepackage{amsthm}
\usepackage{stmaryrd}
\usepackage{graphicx}

%\includegraphics{?} setzt bild ein
%\ref{labelname} erstellt link zu labelname
%\label{labelname} kann einfach irgendwo drangesetz werden

\newtheorem{defi}{Definition}
\newtheorem{axiom}{Axiom}
\newtheorem{nota}{Notation}
\newtheorem{prop}{Proposition}
\newtheorem{satz}{Satz}
\newtheorem{umf}{Umformung}

\newenvironment{beweis}{\par\begingroup%
\settowidth{\leftskip}{\textsc{Beweis:~}}%
\noindent\llap{\textsc{Beweis:~}}}{\hfill$\Box$\par\endgroup}

\renewcommand{\baselinestretch}{1}
\newcommand{\words}{\Sigma^* \backslash \{\epsilon\}}
\newcommand{\etrans}[1]{\bar{\delta}(#1)}
\renewcommand{\P}{\mathbb{P}}

\title{Zettel 10}
\author{Florian Lerch(2404605)/Waldemar Hamm(2410010)}
%\date{} % Activate to display a given date or no date (if empty),
% otherwise the current date is printed 

\begin{document}
\maketitle

\subsection{Aufgabe 30}



\a_1 := X_2 := 0
\a_2 := X_3 := 0
\a_3 := loop X_1(X_2 = X_3 + 1)
[[\a_1]]{(3)}(\a_1,\a_2,\a_3) = [[\a_1 := X_2 := 0]]^{(3)}(\a_1,\a_2,\a_3)
    = (0,\a_2,\a_3)

[[\a_2]]{(3)}(\a_1,\a_2,\a_3) = [[\a_2 := X_3 := 0]]^{(3)}(\a_1,\a_2,\a_3)
    = (\a_1,0,\a_3)

[[\a_3]]{(3)}(\a_1,\a_2,\a_3) = [[loop X_1(X_2 := X_3 + 1)]]^{(3)}(\a_1,\a_2,\a_3)
    = ([[X_2 := X_3 + 1]]^{(3)})^{\a_1}(\a_1,\a_2,\a_3)
    = (\a_1, \a_2 , \a_2 + \a_1)

\subsection{Aufgabe 32}
