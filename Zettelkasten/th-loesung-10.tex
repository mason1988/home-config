
\documentclass[11pt]{amsart}
\usepackage[utf8]{inputenc}
\usepackage{amssymb,amsmath}
\usepackage{verbatim}
\usepackage{color}
\usepackage{geometry}
\geometry{a4paper,left=2cm,right=2cm, top=1.5cm, bottom=1.5cm} 
\usepackage{amsthm}
\usepackage{stmaryrd}
\usepackage{graphicx}

%\includegraphics{?} setzt bild ein
%\ref{labelname} erstellt link zu labelname
%\label{labelname} kann einfach irgendwo drangesetz werden

\newtheorem{defi}{Definition}
\newtheorem{axiom}{Axiom}
\newtheorem{nota}{Notation}
\newtheorem{prop}{Proposition}
\newtheorem{satz}{Satz}
\newtheorem{umf}{Umformung}

\newenvironment{beweis}{\par\begingroup%
\settowidth{\leftskip}{\textsc{Beweis:~}}%
\noindent\llap{\textsc{Beweis:~}}}{\hfill$\Box$\par\endgroup}

\renewcommand{\baselinestretch}{1}
\newcommand{\words}{\Sigma^ \backslash \{\epsilon\}}
\newcommand{\etrans}[1]{\bar{\delta}(#1)}
\renewcommand{\P}{\mathbb{P}}

\title{Zettel 10}
\author{Florian Lerch(2404605)/Waldemar Hamm(2410010)}
%\date{} % Activate to display a given date or no date (if empty),
% otherwise the current date is printed 

\begin{document}
\maketitle

\subsection*{Aufgabe 30}

$A = (Q, \Sigma, \Gamma, \delta, q_0, \bar{b}, F)$ \\
$Q = \{q_0, q_1, q_f \}$ \\
$\Sigma = \{0, 1\}$ \\
$\Gamma = \{0, 1, \bar{b} \}$ \\
$F = \{q_f\}$ \\
$\delta:$ \\
$\begin{array}{cccccc}
q_0 & 0 & 0 & N & q_f & \mbox{ wenn erstes Zeichen 0: 2erkomplement und Binär identisch, also fertig } \\
q_0 & 1 & 1 & R & q_1 & \mbox{ Phase 1: jedes Bit (bis auf das erste) umdrehen } \\
q_1 & 1 & 0 & R & q_1 & \mbox{ Phase 1: jedes Bit (bis auf das erste) umdrehen } \\
q_1 & 0 & 1 & R & q_1 & \mbox{ Phase 1: jedes Bit (bis auf das erste) umdrehen } \\
q_1 & \bar{b} & \bar{b} & L & q_2 & \mbox{ Phase 2: 1 addieren, also von Rechts nach Links } \\
q_2 & 1 & 0 & L & q_2 & \mbox{ jedes Bit umdrehen, bis eine 0 gelesen wird } \\
q_2 & 0 & 1 & N & q_f & \\ 
\end{array}$

\subsection*{Aufgabe 31}

\begin{tabbing}
  $s_n = $\=$ $\underline{in}$ (X_1); $\underline{var}$(X_,X_);$ \\
  \>$ X_2 = 0;$ \\
  \>$ X_3 = 0;$ \\
  \>$ $\underline{loop}$ X_1(X_2 = X_3 + 1);$ \\
  \>$ $\underline{out}$ (X_2);$
\end{tabbing}
\vspace{0.5cm}
$\alpha_1 := X_2 := 0$ \\
$\alpha_2 := X_3 := 0$ \\
$\alpha_3 := \underline{loop} X_1(X_2 = X_3 + 1)$ \\
\begin{tabbing}

$[[\alpha_1]]^{(3)}(\alpha_1,\alpha_2,\alpha_3)$\=$ = [[\alpha_1 := X_2 := 0]]^{(3)}(\alpha_1,\alpha_2,\alpha_3)$ \\
$ $\>$   = (\alpha_1,0,\alpha_3)$ \\[0.3cm]
$[[\alpha_2]]^{(3)}(\alpha_1,\alpha_2,\alpha_3) = [[\alpha_2 := X_3 := 0]]^{(3)}(\alpha_1,\alpha_2,\alpha_3)$ \\
$  $\>$  = (\alpha_1,\alpha_2,0)$ \\[0.3cm]
$[[\alpha_3]]^{(3)}(\alpha_1,\alpha_2,\alpha_3) = [[\underline{loop} X_1(X_2 := X_3 + 1)]]^{(3)}(\alpha_1,\alpha_2,\alpha_3)$ \\
$   $\>$ = ([[X_2 := X_3 + 1]]^{(3)})^{\alpha_1}(\alpha_1,\alpha_2,\alpha_3)$ \\
% \> $ = ( \alpha_1 , \alpha_3^{\alpha_1} + 1^{\alpha_1}, \alpha_3)(\alpha_1,\alpha_2,\alpha_3)$ \\[0.5cm]
Für $\alpha_1 > 0$: \\
\> $ = ( \alpha_1 , \alpha_3 + 1, \alpha_3)(\alpha_1,\alpha_2,\alpha_3)$ \\
Für $\alpha_1 = 0$: \\
\> $ = ( \alpha_1 , \alpha_3, \alpha_3)(\alpha_1,\alpha_2,\alpha_3)$ \\
% wie soll man denn schleifen einsetzen, die eins auf einen _anderen_ Wert addieren?
\end{tabbing}

\begin{tabbing}
$[[S_n]](\alpha_1)$ \= $= (\underline{out}^{(3)}_2 \circ [[\alpha_3]]^{(3)}  
\circ [[\alpha_2]]^{(3)} \circ [[\alpha_1]]^{(3)} \circ \underline{in}^{(1)}_3)(\alpha_1)$ \\
\> $= (\underline{out}^{(3)}_2 \circ [[\alpha_3]] \circ [[\alpha_2]]^{(3)} \circ [[\alpha_1]]^{(3)})(\alpha_1,0,0)$ \\
\> $= (\underline{out}^{(3)}_2 \circ [[\alpha_3]] \circ [[\alpha_2]]^{(3)})(\alpha_1,0,0)$ \\
\> $= (\underline{out}^{(3)}_2 \circ [[\alpha_3]])(\alpha_1,0,0)$ \\
%\> $= (\underline{out}^{(3)}_1)(\alpha_1,0^{\alpha_1} + 1,0)$ \\
%\> $= (\alpha_1,1,0) = $ \\
Für $\alpha_1 > 0$: \\
\> $= (\underline{out}^{(3)}_2)(\alpha_1,1,0) = 1$ \\
Für $\alpha_1 = 0$: \\
\> $= (\underline{out}^{(3)}_2)(\alpha_1,0,0) = 0$ \\
% hier  ist dann immernoch das Problem mit der Schleife
$\Rightarrow [[S_n]]: \mathbb{N} \rightarrow \mathbb{N}, (\alpha_1) \rightarrow \begin{cases} 1 &\mbox{falls } \alpha_1>0 \\ 0 &\mbox{sonst} \end{cases}$
\end{tabbing}

\subsection*{Aufgabe 32}

\subsubsection*{a)} 
Es gilt: $power(a,0) = 1 = C_0^{(1)}(a) + 1 = S(C_0^{(1)}(a)) = 
S \circ [C_0^{(1)}](a)$ \\
und $power(a,b+1) = power(a,b)*a = mult(power(a,b),a) =
mult \circ [p_3^{(3)},p_1^{(3)}](a,b,power(a,b))$ \\
Dies entspricht dem Schema mit $f = \circ [C_0^{(1)}]$ und $g = mult \circ [p_3^{(3)},p_1^{(3)}]$ \\

\subsubsection*{b)}
Es gilt: $anz(0) = 1 = C_0^{(0)}() + 1 = S(C_0^{(0)}()) = 
S \circ [C_0^{(0)}]()$ \\
Dies entspricht im Schema $f = S \circ [C_0]^{(0)}]$ \\
Ferner gilt: $anz(n+1) = 4^n - anz(n) + anz(n)  3 = 4^n + anz(n)  2$ \\
$= power(C_4^{(1)}(n),n) - mul(anz(n),C_3^{(0)}()) = sub(power(C_4^{(0)}(),n),mul(anz(n),C_3^{(0)}()))$ \\

\end{document}
