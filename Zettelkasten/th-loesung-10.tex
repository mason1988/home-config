
\documentclass[11pt]{amsart}
\usepackage[utf8]{inputenc}
\usepackage{amssymb,amsmath}
\usepackage{verbatim}
\usepackage{color}
\usepackage{geometry}
\geometry{a4paper,left=2cm,right=2cm, top=1.5cm, bottom=1.5cm} 
\usepackage{amsthm}
\usepackage{stmaryrd}
\usepackage{graphicx}

%\includegraphics{?} setzt bild ein
%\ref{labelname} erstellt link zu labelname
%\label{labelname} kann einfach irgendwo drangesetz werden

\newtheorem{defi}{Definition}
\newtheorem{axiom}{Axiom}
\newtheorem{nota}{Notation}
\newtheorem{prop}{Proposition}
\newtheorem{satz}{Satz}
\newtheorem{umf}{Umformung}

\newenvironment{beweis}{\par\begingroup%
\settowidth{\leftskip}{\textsc{Beweis:~}}%
\noindent\llap{\textsc{Beweis:~}}}{\hfill$\Box$\par\endgroup}

\renewcommand{\baselinestretch}{1}
\newcommand{\words}{\Sigma^* \backslash \{\epsilon\}}
\newcommand{\etrans}[1]{\bar{\delta}(#1)}
\renewcommand{\P}{\mathbb{P}}

\title{Zettel 10}
\author{Florian Lerch(2404605)/Waldemar Hamm(2410010)}
%\date{} % Activate to display a given date or no date (if empty),
% otherwise the current date is printed 

\begin{document}
\maketitle

\subsection*{Aufgabe 30}

$A = (Q, \Sigma, \Gamma, \delta, q_0, \bar{b}, F)$ \\
$Q = \{q_0, q_1, q_f \}$ \\
$\Sigma = \{0, 1\}$ \\
$\Gamma = \{0, 1, \bar{b} \}$ \\
$F = \{q_f\}$ \\
$\delta:$ \\
$\begin{array}{cccccc}
q_0 & 0 & 0 & N & q_f & \mbox{ wenn erstes Zeichen 0: 2erkomplement und Binär identisch, also fertig } \\
q_0 & 1 & 1 & R & q_1 & \mbox{ Phase 1: jedes Bit (bis auf das erste) umdrehen } \\
q_1 & 1 & 0 & R & q_1 & \mbox{ Phase 1: jedes Bit (bis auf das erste) umdrehen } \\
q_1 & 0 & 1 & R & q_1 & \mbox{ Phase 1: jedes Bit (bis auf das erste) umdrehen } \\
q_1 & \bar{b} & \bar{b} & L & q_2 & \mbox{ Phase 2: 1 addieren, also von Rechts nach Links } \\
q_2 & 1 & 0 & L & q_2 & \mbox{ jedes Bit umdrehen, bis eine 0 gelesen wird } \\
q_2 & 0 & 1 & N & q_f & \\ 
\end{array}$

\subsection*{Aufgabe 31}

\begin{tabbing}
  $s_n = $\=$ $\underline{in}$ (X_1); $\underline{var}$(X_,X_);$ \\
  \>$ X_2 = 0;$ \\
  \>$ X_3 = 0;$ \\
  \>$ $\underline{loop}$ X_1(X_2 = X_3 + 1);$ \\
  \>$ $\underline{out}$ (X_2);$
\end{tabbing}
  
  \vspace{0.5cm}
  
  $\alpha_1 := X_2 := 0$ \\
  $\alpha_2 := X_3 := 0$ \\
  $\alpha_3 := loop X_1(X_2 = X_3 + 1)$ \\
  $[[\alpha_1]]{(3)}(\alpha_1,\alpha_2,\alpha_3) = [[\alpha_1 := X_2 := 0]]^{(3)}(\alpha_1,\alpha_2,\alpha_3)$ \\
  $    = (0,\alpha_2,\alpha_3)$ \\[0.3cm]
  
  $[[\alpha_2]]{(3)}(\alpha_1,\alpha_2,\alpha_3) = [[\alpha_2 := X_3 := 0]]^{(3)}(\alpha_1,\alpha_2,\alpha_3)$ \\
  $    = (\alpha_1,0,\alpha_3)$ \\[0.3cm]
  
  $[[\alpha_3]]{(3)}(\alpha_1,\alpha_2,\alpha_3) = [[loop X_1(X_2 := X_3 + 1)]]^{(3)}(\alpha_1,\alpha_2,\alpha_3)$ \\
  $    = ([[X_2 := X_3 + 1]]^{(3)})^{\alpha_1}(\alpha_1,\alpha_2,\alpha_3)$ \\
  $    = (\alpha_1, \alpha_2 , \alpha_2 + \alpha_1) $ \\[0.5cm]

\subsection{Aufgabe 32}

\end{document}
