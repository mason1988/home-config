% Created 2013-02-10 So 18:28
\documentclass[11pt]{article}
\usepackage[utf8]{inputenc}
\usepackage[T1]{fontenc}
\usepackage{fixltx2e}
\usepackage{graphicx}
\usepackage{longtable}
\usepackage{float}
\usepackage{wrapfig}
\usepackage{soul}
\usepackage{textcomp}
\usepackage{marvosym}
\usepackage{wasysym}
\usepackage{latexsym}
\usepackage{amssymb}
\usepackage{hyperref}
\tolerance=1000
\providecommand{\alert}[1]{\textbf{#1}}

\title{Übung 12}
\author{florian,waldemar,rene}
\date{2013-01-17 Do}
\hypersetup{
  pdfkeywords={},
  pdfsubject={},
  pdfcreator={Emacs Org-mode version 7.8.11}}

\begin{document}

\maketitle

\setcounter{tocdepth}{3}
\tableofcontents
\vspace*{1cm}
\section{organisation}
\label{sec-1}

\begin{itemize}
\item email: selecture@mathematik.uni-marburg.de
\item Gruppe 1
\end{itemize}
\section{Übung 11}
\label{sec-2}
\subsection{Aufgabe A}
\label{sec-2-1}
\subsubsection{Lösung}
\label{sec-2-1-1}

Erst durch die Anforderungsanalyse kann der bzw. können die Programmierer erst wirklich
in Erfahrung bringen, was der Kunde wirklich haben möchte. Sollte man
vorher schon anfangen zu entwickeln, so besteht das Risiko, dass man
an dem vorbei entwickelt was der Kunde haben möchte. Vielleicht
investiert man viel Zeit in Features an denen der Kunde eigentlich gar
kein richtiges Interesse hat, oder aber trifft bereits
Designentscheidungen, die im Widerspruch zum eigentlichen Wunsch des
Kunden stehen.
Aber es können sich auch technischere Probleme ergeben. Durch die
Anforderungsanalyse wird erst effektive Planung und Zusammenarbeit
mehrerer Programmierer möglich. Wenn nicht klar definiert ist, was das
Programm alles leisten können muss, ist effektives Planung der Arbeit
nur schwer möglich so dass z.B. echte Überprüfbarkeit der Arbeit gar
nicht machbar ist, was dann natürlich auch wieder zu Problemen mit dem
Kunden führen kann, da zu diesem Zeitpunkt in der Regel ja auch noch
kein richtiger Vertrag besteht.
\subsection{Aufgabe B}
\label{sec-2-2}
\subsubsection{Lösung}
\label{sec-2-2-1}

\begin{itemize}
\item wo liegt die Datenbank, mit welcher die Route ermittelt wird?
\item welche Anbindung an andere Systeme besteht, um die Kreditkarte
      zu prüfen?
\end{itemize}
\subsection{Aufgabe C}
\label{sec-2-3}
\subsubsection{Lösung}
\label{sec-2-3-1}
\begin{itemize}

\item Fahrschein kaufen
\label{sec-2-3-1-1}%
\begin{itemize}

\item Titel\\
\label{sec-2-3-1-1-1}%
Fahrschein kaufen

\item Kurzbeschreibung\\
\label{sec-2-3-1-1-2}%
Der Kunde betätigt den Startknopf, wählt dann seine Route und
      und bezahlt dann mit einer Kreditkarte.

\item Aktoren\\
\label{sec-2-3-1-1-3}%
Potenzielle Fahrgast

\item Vorbedingungen\\
\label{sec-2-3-1-1-4}%
keine

\item Beschreibung Ablauf
\label{sec-2-3-1-1-5}%
\begin{enumerate}
\item Der Kunde betätigt den Startknopf
\item Automat zeigt Menü für Zielbahnhofsuche an
\item Kunde gibt Zielbahnhof an
\item Automat liest Suchergebnisse aus der Datenbank
\item Kunde bestätigt eins der Ergebnisse
\item Automat berechnet Preis
\item Kunde gibt Kreditkarte ein
\item Kunde gibt Geheimnummer ein
\item Automat prüft Nummer und Kreditkarte
\item falls Erfolg: Automat hebt Geld ab, druckt Fahrschein und
          gibt Karte wieder aus
\end{enumerate}

\item Auswirkungen\\
\label{sec-2-3-1-1-6}%
Unter Umständen gibt es eine maximale Fahrgastanzahl für die
      Züge welcher sich dann genähert wurde. Allgemein endet der
      Automat aber im selben Zustand, in welchem er angefangen hat,
      also ohne echte Auswirkungen.

\item Anmerkungen
\label{sec-2-3-1-1-7}%
\end{itemize} % ends low level

\item Rechtschreibprüfung in einem Texteditor
\label{sec-2-3-1-2}%
\begin{itemize}

\item Titel\\
\label{sec-2-3-1-2-1}%
Rechtschreibprüfung in einem Texteditor

\item Kurzbeschreibung\\
\label{sec-2-3-1-2-2}%
Benutzer startet die Rechtschreibprüfung und bekommt dann alle
      Fehler angezeigt und entscheidet jeweils, was pasieren soll.

\item Aktoren\\
\label{sec-2-3-1-2-3}%
Benutzer des Texteditors

\item Vorbedingungen\\
\label{sec-2-3-1-2-4}%
Texteditor mit Text drin muss geöffnet sein.

\item Beschreibung Ablauf
\label{sec-2-3-1-2-5}%
\begin{enumerate}
\item Benutzer startet Rechtschreibprüfung
\item Programm vergleicht Wortweise die Inhalte des Textdokumentes mit Wortdatenbank
         und bildet eine Liste mit Wörtern die nicht gefunden werden können
\item Es wird ein Fenster eingeblendet in welchem jeweils der aktuelle Fehler angezeigt wird
         zusammen mit ähnlichen Worteinträgen aus der Datenbank, außerdem ein Button fürs ignorieren
\item der Benutzer wählt entwededer einen Worteintrag oder ignorieren
\item das Programm ersetzt das Wort und springt weiter zum nächsten Fehler, oder springt direkt
         weiter
\end{enumerate}

\item Auswirkungen\\
\label{sec-2-3-1-2-6}%
Die Fehler im Text wurden korrigiert.

\item Anmerkungen
\label{sec-2-3-1-2-7}%
\end{itemize} % ends low level
\end{itemize} % ends low level
\subsection{Aufgabe D}
\label{sec-2-4}
\subsubsection{Lösung}
\label{sec-2-4-1}

\begin{enumerate}
\item Die Ladezeiten zwischen den einzelnen Menüeinträgen muss in jeder Situation unterhalb von 2 Sekunden liegen.
\item Die Software darf mitsamt der Datenbank und allen zur Funktion benötigten Modulen nicht größer als 20mb werden.
\item Die Menüs müssen mittels eines Touchpads bedienbar sein, wobei die Fehlerrate nicht oder falsch erkannter
    Eingaben im Durchschnitt unterhalb von 5\% liegen muss.
\item Die Software muss eine Fehlerfreie und abgesicherte Kommunikation mit den Bankservern garantieren
\item Das System darf innerhalb eines Monats bei vielfacher Nutzung nur maximal ein mal abstürzen. Im Falle eines
   absturzes muss garantiert sein, dass der Nutzer keine Möglichkeit der Interaktion mehr besitzt.
\end{enumerate}

\end{document}