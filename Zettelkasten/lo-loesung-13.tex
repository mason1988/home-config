
\documentclass[11pt]{amsart}
  \usepackage[utf8]{inputenc}
  \usepackage{amssymb,amsmath}
  \usepackage{verbatim}
  \usepackage{color}
  \usepackage{geometry}
  \usepackage{amsthm}
  \usepackage{stmaryrd}
  \usepackage{graphicx}
%-----------
\renewcommand{\baselinestretch}{2}

\newtheorem{defi}{Definition}
\newtheorem{axiom}{Axiom}
\newtheorem{nota}{Notation}
\newtheorem{prop}{Proposition}
\newtheorem{satz}{Satz}
\newtheorem{umf}{Umformung}

\newcommand{\words}{\Sigma^* \backslash \{\epsilon\}}
\newcommand{\etrans}[1]{\bar{\delta}(#1)}

\newenvironment{beweis}{\par\begingroup%
\settowidth{\leftskip}{\textsc{Beweis:~}}%
\noindent\llap{\textsc{Beweis:~}}}{\hfill$\Box$\par\endgroup}
%----------
\renewcommand{\P}{\mathbb{P}}
\renewcommand{\a}{\alpha}
\renewcommand{\,}{\quad}
\renewcommand{\o}{\omega}
\renewcommand{\O}{\Omega}
\newcommand{\s}{\sigma}
\renewcommand{\S}{\Sigma}
\renewcommand{\lim}[1]{\underset{#1}{lim}}
\newcommand{\m}[1]{\mbox{#1}}
\newcommand{\N}{\mathbb{N}}
\newcommand{\R}{\mathbb{R}}
\newcommand{\E}{\exists}
\newcommand{\A}{\forall}
\newcommand{\fr}[2]{\frac{#1}{#2}}
\newcommand{\ri}{\rightarrow}
\newcommand{\lori}{\longrightarrow}
\newcommand{\Ri}{\Rightarrow}
\newcommand{\Lri}{\Leftrightarrow}
\newcommand{\tri}{\mapsto} %tail right arrow
\newcommand{\ov}[2]{\overset{#1}{#2}}
\newcommand{\un}[2]{\underset{#1}{#2}}
\newcommand{\e}{\epsilon}
\newcommand{\busset}{\supset}
\newcommand{\inc}{\hspace*{0.5cm}}
\newcommand{\dinc}{\hspace*{1.2cm}}
\newcommand{\zz}{\overset{!}{=}}
\geometry{a4paper,left=2cm,right=2cm, top=1.5cm, bottom=1.5cm} 
\title{Zettel 13 }
\author{Florian Lerch(2404605)/Waldemar Hamm(2410010)}
\begin{document}
\maketitle

\section*{Aufgabe 42}
\subsection*{a)}.\\
Sei $X_s$ das Supremum $\Lri \A n \in \N: \fr{X_n}{X_s} \Ri \A n \in N: [X_n] \leq [X_s]$ \\
Sei $X_i$ das Infinum $\Lri \A n \in \N: \fr{X_i}{X_n} \Ri \A n \in N: [X_i] \leq [X_n]$ \\
\subsection*{b)}.\\
Sei erneut $X_s$ das Supremum und $X_i$ das Infinum. \\
$\N \bs \V_nX_n = \N \bs [X_s] = $ %keine Ahnung wie's hier weitergeht

\section*{Aufgabe 43}
\subsection*{a)}.\\ % FIXME charakter nachweis fehlt
Sei $\chi \V_{n=0}^{N}X_n = 1  \Lri \chi(sup_{n\in N}X_n) = 1$ $\Lri n_0 \in (sup_{n\in N}X_n)$ $\Lri \E i \in \N: \chi{X_i} = 1 \Lri \V^{N}_{n=0}\chi(X_n)$ $=1$ \\
Sei $\chi \W_{n=0}^{N}X_n = 1  \Lri \chi(inf_{n\in N}X_n) = 1$ $\Lri n_0 \in (inf_{n\in N}X_n)$ $\Lri \A i \in \N: \chi{X_i} = 1 \Lri \W^{N}_{n=0}\chi(X_n)$ $=1$ \\
\subsection*{b)}.\\ % naja...
Im Beweis für a) wurde nichts vorrausgesetzt, was nicht allgemein für abzählbare Familien gelten würde, also ja.

\section*{Aufgabe 44}
\subsection*{a)}.\\

\end{document}
