
\documentclass[11pt]{amsart}
  \usepackage[utf8]{inputenc}
  \usepackage{amssymb,amsmath}
  \usepackage{verbatim}
  \usepackage{color}
  \usepackage{geometry}
  \usepackage{amsthm}
  \usepackage{stmaryrd}
  \usepackage{graphicx}
%-----------
\renewcommand{\baselinestretch}{2}

\newtheorem{defi}{Definition}
\newtheorem{axiom}{Axiom}
\newtheorem{nota}{Notation}
\newtheorem{prop}{Proposition}
\newtheorem{satz}{Satz}
\newtheorem{umf}{Umformung}

\newcommand{\words}{\Sigma^* \backslash \{\epsilon\}}
\newcommand{\etrans}[1]{\bar{\delta}(#1)}

\newenvironment{beweis}{\par\begingroup%
\settowidth{\leftskip}{\textsc{Beweis:~}}%
\noindent\llap{\textsc{Beweis:~}}}{\hfill$\Box$\par\endgroup}
%----------
\renewcommand{\P}{\mathbb{P}}
\renewcommand{\a}{\alpha}
\renewcommand{\,}{\quad}
\newcommand{\m}[1]{\mbox{#1}}
\newcommand{\N}{\mathbb{N}}
\newcommand{\R}{\mathbb{R}}
\newcommand{\E}{\exists}
\newcommand{\A}{\forall}
\newcommand{\fr}[2]{\frac{#1}{#2}}
\newcommand{\ri}{\rightarrow}
\newcommand{\Ri}{\Rightarrow}
\newcommand{\Lri}{\Leftrightarrow}
\newcommand{\tri}{\mapsto} %tail right arrow
\newcommand{\ov}[2]{\overset{#1}{#2}}
\newcommand{\un}[2]{\underset{#1}{#2}}
\newcommand{\busset}{\supset}
\newcommand{\inc}{\hspace*{0.5cm}}
\newcommand{\dinc}{\hspace*{1.2cm}}
\newcommand{\zz}{\overset{!}{=}}
\geometry{a4paper,left=2cm,right=2cm, top=1.5cm, bottom=1.5cm} 
\title{Zettel 12 }
\author{Florian Lerch(2404605)/Waldemar Hamm(2410010)}
\begin{document}
\maketitle

\section*{40)}
\subsection*{a)}
$z = (x_0)_{\g}^{|\W_{x_3}A|} = (x_0)_{\g}^{\{x_0,x_1,x_2,x_3,x_4\}}$ \\ % hier wurde x_3 wegen der Belegung aus |A| rausgenommen
% bei einem A kommt ein Term raus, bei \W entstehen dann mehrere terme
% aber das |A| schmeißt dann wieder alles zusammen
% als Trägermenge werden daher nur die Variablen x_0 bis x_2 betrachtet, da x_3 durch seine Belegung ja
% im Grunde nichtmehr existiert, andererseits kommt x_3 schon in der Formel vor, und in der Substitution
% geht es sogar bis hoch zu x_4 / jetzt doch wieder drin
% unter umständen entspricht das x_3 am quantor tatsächlich allen möglichen substitutionen
$\Ri \m{Ausnahmemenge} = |\g((\{x_0,x_1,x_2,x_3,x_4\} \bs \{x_0\})^{\g})| = |\g(\{x_1,x_2,x_3,x_4\}^{\g})|$ \\
$= |\g\{x_1\}|$ % nur das was sich verändert
$= |\{\tilde s_2x_4x_4\}|$ % Anwendung 
$= \{x_4\}$ \\
$x_0 \not \in \{x_4\} \Ri z = x_0$ \\
$\Ri (x_0)_{\g}^{|\W_{x_3}A|} = x_0$
\subsection*{b)}
$\eta = \g^z_{x_0} = \iota_{x_0|x_1}^{z|\tilde s_2 x_4x_4}$ %einfach nur substitution von x_0 eingesetzt
\subsection*{c)}
$\o = (x_3)^{|A|}_{\eta} = (x_3)^{\{x_0,x_1,x_2,x_3,x_4\}}$
$\Ri \m{Ausnahmemenge} = |\g((\{x_0,x_1,x_2,x_3,x_4\} \bs \{x_3\})^{\g})| = |\g((\{x_0,x_1,x_2,x_4\})^{\g})|$ \\
$= |\g(\{x_0,x_1\})| = |\{z, \tilde s_2 x_4 x_4\}| = \{z,x_4\}$ \\
$x_3 \not \in \{z,x_4\} \Ri \o = x_3$ \\
$\Ri (x_3)^{|A|}_{\eta} = x_3$
\subsection*{d)}
$\eta_{x_3}^{\o} = \iota_{x_0|x_1|x_3}^{z|\tilde s_2x_4 x_4|\o}$ \\
\subsection*{e)}
$\eta \circ \W_{x_3}A = \W_{x_3} \eta^{x_3}_{x_3} \circ A$ \\
$\eta^{x_3}_{x_3} \Ri \i^{z|\tilde s_2x_4x_4}_{x_0|x_1}$ \\
$\Ri (\i_{x_0|x_1|x_3}^{z|\tilde s_2x_4 x_4|\o}) \circ (\W_{x_3}A)$ \\
$= \W_{x_3}\i^{z|\tilde s_2x_4x_4}_{x_0|x_1} \circ A$ \\
$= \W_{x_3}p_2(\i^{z|\tilde s_2x_4x_4}_{x_0|x_1}s_2x_0x_1)(\i^{z|\tilde s_2x_4x_4}_{x_0|x_1}\tilde s_2 x_2 s_1 x_3)$ \\
$= x_0^{\v}p_2s_2z\tilde s_2x_4x_4 \tilde s_2x_2s_1x_3$ \\
\subsection*{f)}
$\g \circ \V_{x_0}\W_{x_3}A = \V_{x_0}\W_{x_3} \g^{x_0}_{x_0}$ \\
$\g^{x_0}_{x_0} \Ri \i^{\tilde s_2 x_4 x_4}_{x_1}$ \\
$\Ri (\i^{z|\tilde s_2 x_4 x_4}_{x_0 | x_1} \circ (\V_{x_0}(\W_{x_3}A))$ \\
$= \V_{x_0} \i^{\tilde s_2 x_4 x_4}_{x_1} \circ (x_0^{\v}p_2s_2z\tilde s_2 x_4 x_4 \tilde s_2 x_2 s_1 x_3)$ \\

\section*{41)}
\subsection*{a)}
$(\a^{\v} \circ \g) (x_j) = \begin{cases} 4*4 = 16 &\mbox{falls } j=1 \\ 3j &\mbox{sonst} \end{cases}$

\end{document}
